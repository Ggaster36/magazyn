%% Generated by Sphinx.
\def\sphinxdocclass{report}
\documentclass[letterpaper,10pt,polish]{sphinxmanual}
\ifdefined\pdfpxdimen
   \let\sphinxpxdimen\pdfpxdimen\else\newdimen\sphinxpxdimen
\fi \sphinxpxdimen=.75bp\relax
\ifdefined\pdfimageresolution
    \pdfimageresolution= \numexpr \dimexpr1in\relax/\sphinxpxdimen\relax
\fi
%% let collapsible pdf bookmarks panel have high depth per default
\PassOptionsToPackage{bookmarksdepth=5}{hyperref}

\PassOptionsToPackage{booktabs}{sphinx}
\PassOptionsToPackage{colorrows}{sphinx}

\PassOptionsToPackage{warn}{textcomp}
\usepackage[utf8]{inputenc}
\ifdefined\DeclareUnicodeCharacter
% support both utf8 and utf8x syntaxes
  \ifdefined\DeclareUnicodeCharacterAsOptional
    \def\sphinxDUC#1{\DeclareUnicodeCharacter{"#1}}
  \else
    \let\sphinxDUC\DeclareUnicodeCharacter
  \fi
  \sphinxDUC{00A0}{\nobreakspace}
  \sphinxDUC{2500}{\sphinxunichar{2500}}
  \sphinxDUC{2502}{\sphinxunichar{2502}}
  \sphinxDUC{2514}{\sphinxunichar{2514}}
  \sphinxDUC{251C}{\sphinxunichar{251C}}
  \sphinxDUC{2572}{\textbackslash}
\fi
\usepackage{cmap}
\usepackage[T1]{fontenc}
\usepackage{amsmath,amssymb,amstext}
\usepackage{babel}



\usepackage{tgtermes}
\usepackage{tgheros}
\renewcommand{\ttdefault}{txtt}



\usepackage[Sonny]{fncychap}
\ChNameVar{\Large\normalfont\sffamily}
\ChTitleVar{\Large\normalfont\sffamily}
\usepackage{sphinx}

\fvset{fontsize=auto}
\usepackage{geometry}


% Include hyperref last.
\usepackage{hyperref}
% Fix anchor placement for figures with captions.
\usepackage{hypcap}% it must be loaded after hyperref.
% Set up styles of URL: it should be placed after hyperref.
\urlstyle{same}

\addto\captionspolish{\renewcommand{\contentsname}{Contents:}}

\usepackage{sphinxmessages}
\setcounter{tocdepth}{1}



\title{zadanie\_laby1}
\date{04 lis 2025}
\release{1}
\author{290306}
\newcommand{\sphinxlogo}{\vbox{}}
\renewcommand{\releasename}{Wydanie}
\makeindex
\begin{document}

\ifdefined\shorthandoff
  \ifnum\catcode`\=\string=\active\shorthandoff{=}\fi
  \ifnum\catcode`\"=\active\shorthandoff{"}\fi
\fi

\pagestyle{empty}
\sphinxmaketitle
\pagestyle{plain}
\sphinxtableofcontents
\pagestyle{normal}
\phantomsection\label{\detokenize{index::doc}}


\sphinxAtStartPar
Add your content using \sphinxcode{\sphinxupquote{reStructuredText}} syntax. See the
\sphinxhref{https://www.sphinx-doc.org/en/master/usage/restructuredtext/index.html}{reStructuredText}
documentation for details.

\sphinxstepscope


\chapter{Wprowadzenie}
\label{\detokenize{rozdzial1/index:wprowadzenie}}\label{\detokenize{rozdzial1/index::doc}}
\sphinxAtStartPar
Jest to strona o bigosie nie poważna stona o bigosie jest tu wiele razy skopiowany opis bigosu z pana tadeusza tabelka z tematyką bigosu,
lista z tematyką bigosu oraz zdjęcia kotków bez powiązania z bigosem. Podsumuwując bigos i kotki to jest ta strona.


\section{BIGOS}
\label{\detokenize{rozdzial1/index:bigos}}
\sphinxstepscope


\chapter{Tabelka i Lista}
\label{\detokenize{rozdzial2/index:tabelka-i-lista}}\label{\detokenize{rozdzial2/index::doc}}
\sphinxAtStartPar
tutaj jest tabelka i lista oczywiście z tematyką bigosu


\section{Tabelka}
\label{\detokenize{rozdzial2/index:tabelka}}

\begin{savenotes}\sphinxattablestart
\sphinxthistablewithglobalstyle
\centering
\begin{tabulary}{\linewidth}[t]{TTT}
\sphinxtoprule
\sphinxstyletheadfamily 
\sphinxAtStartPar
TYP Bigosu
&\sphinxstyletheadfamily 
\sphinxAtStartPar
smak bigosu
&\sphinxstyletheadfamily 
\sphinxAtStartPar
zdobycie
\\
\sphinxmidrule
\sphinxtableatstartofbodyhook
\sphinxAtStartPar
Bigos normalny
&
\sphinxAtStartPar
smaczny
&
\sphinxAtStartPar
sklep
\\
\sphinxhline
\sphinxAtStartPar
bigos dziwny
&
\sphinxAtStartPar
fioletowy
&
\sphinxAtStartPar
alejka 8
\\
\sphinxhline
\sphinxAtStartPar
bigos kolorowy
&
\sphinxAtStartPar
magiczny
&
\sphinxAtStartPar
pustka
\\
\sphinxbottomrule
\end{tabulary}
\sphinxtableafterendhook\par
\sphinxattableend\end{savenotes}


\section{lista}
\label{\detokenize{rozdzial2/index:lista}}\begin{description}
\sphinxlineitem{\#.Powody aby lubić bigos}
\sphinxAtStartPar
\#.jest smaczny
\#.jest zdrowy
\#.da się zrobić go w dużym garze
\#.jest bigosem
\#.potwór spagethi przyjaźni się z potworem bigos
\#.nawet cthulu budzi się żeby zjeść bigos

\end{description}

\sphinxstepscope


\chapter{Obrazy}
\label{\detokenize{rozdzial3/index:obrazy}}\label{\detokenize{rozdzial3/index::doc}}


\renewcommand{\indexname}{Indeks}
\printindex
\end{document}